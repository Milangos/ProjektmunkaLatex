\documentclass[a4paper,12pt]{article}

\usepackage[utf8]{inputenc}
\usepackage[T1]{fontenc}
\usepackage[hungarian]{babel}
\usepackage{graphicx}
\usepackage{amsmath, amssymb}
\usepackage{hyperref}
\usepackage{setspace}
\usepackage{indentfirst}
\usepackage{array}
\usepackage{tabularx}
\usepackage{makecell}
\setlength{\parindent}{12pt}
\onehalfspacing

\title{Vércukorszint dinamikák klaszterezése\\1-es típusú cukorbetegségben}
\author{}
\date{}

\begin{document}

\maketitle

\section*{1-es típusú cukorbetegség}

Az 1-es típusú diabetes mellitus (más néven inzulinfüggő diabetes mellitus) egy autoimmun betegség, amelyet abszolút inzulinhiány okoz. Hátterében az áll, hogy a szervezet immunrendszere idegenként ismeri fel a saját sejtek egy részét, és autoimmun gyulladás következtében elpusztulnak a hasnyálmirigy inzulint termelő béta-sejtjei. Az inzulin abszolút hiánya és a hipoglikémiára adott elégtelen válasz miatt a vércukorszint gyakran megjósolhatatlanul és szabálytalanul ingadozik. Ez a diabétesz bármely életkorban előfordulhat, de leggyakrabban gyermek- és fiatal felnőttkorban jelentkezik.

Korábban egészséges embereket támad meg, normál testsúllyal. A betegek általában soványak, gyakori a jelentős fogyás a betegség megállapítása előtt. A tünetek gyorsan alakulnak ki, a betegek kezeléséhez inzulin szükséges. Kezeletlen vagy rosszul kezelt esetben ketoacidózisos kóma alakulhat ki.

Az 1-es típusú diabétesz általában a diabéteszes esetek 10\%-át teszi ki Európában és Amerikában. Leggyakoribb Skandináviában, míg ritka Japánban és Koreában. Európán belül is jelentősen változó gyakoriságú, több mint tízszer gyakoribb Finnországban, mint Macedóniában. Magyarországon is egyre nő ennek a típusnak a gyakorisága.

A korábbi fiatalkori cukorbetegség elnevezést ma már nem használják, ugyanis bármely életkorban elkezdődhet. Az 1-es típusú diabétesz kialakulását összefüggésbe hozzák a csecsemőkori tehéntejfogyasztással. Egy svéd diabetológus csoport szerint hatékonyabban kezelhető alacsony szénhidrát étrenddel, mint a jelenlegi magas szénhidrát alapú protokollal.

Jelenleg nincs olyan eljárás, amely meggyógyítaná a cukorbetegséget. Számos kísérlet irányul az inzulintermelés helyreállítására, de tartós eredmény nincs. A kutatás érinti a mesterséges hasnyálmirigy rendszereket és védett béta-sejtköteg beültetést is.

Az 1-es típus nehezebben tartható karban, mint a 2-es. Gyakori az inzulinszint indokolatlan változása, ami növeli a ketoacidózis kockázatát. Egyéb szövődmények: fertőzések, gyomorbénulás, hormonproblémák (pl. Addison-kór). Genetikai tényezők növelhetik a hajlamot, például bizonyos HLA genotípusok.

Táplálkozási tényezők és vírusok szerepét is vizsgálják, de bizonyítékok nem teljesek. A gluténben található gliadin kapcsolatba hozható, emiatt gyakoribb lehet a lisztérzékenység.

A betegség bármely életkorban kezdődhet, gyakran felnőttkorban fedezik fel. Ezt a formát LADA-nak (Latent Autoimmune Diabetes in Adults) nevezik.

\section*{1-es típusú cukorbetegség kezelése}

A kezelés azonnali inzulinnal kezdődik. Az inzulin bevihető injekcióval, inzulintollal vagy inzulinpumpával. A kezelés mellett szükséges a rendszeres vércukormérés (napi 4–12 alkalom). Ez történhet ujjbegyszúrással vagy folyamatos vércukormonitorozással (CGM).

A legjobb eredmény CGM + inzulinpumpa kombinációjával érhető el.

Alacsony szénhidrát étrenddel néhány esetben javulásról számoltak be. Magyarországon dr. Tóth Csaba paleo-ketogén étrenddel kezelt betegeknél inzulintermelés visszatérését dokumentálták.

\section*{Szimulátor 1: py-mGIPsim}

Az mGIPsim (Metabolic Glucose–Insulin Physiology Simulator) egy nyílt forráskódú metabolikus szimulátor, amely 20 virtuális, 1-es típusú cukorbeteg páciens vércukor- és inzulinszintjét modellezi különféle terápiás, étkezési és fizikai aktivitási körülmények között.

Python nyelven készült, vektorizált JIT-fordítással optimalizált számítási modellel. Moduláris felépítésű, bővíthető új modellekkel és algoritmusokkal.

Felhasználási módok:

\begin{itemize}
    \item Streamlit alapú grafikus felület,
    \item Interaktív parancssori mód,
    \item Egyszerű CLI mód egy parancssorból.
\end{itemize}

Támogatott terápiák:

\begin{itemize}
    \item MDI (Multiple Daily Injections),
    \item SAP (Sensor-Augmented Pump Therapy),
    \item Hybrid Closed-Loop rendszerek.
\end{itemize}

A szimuláció képes étkezések, mozgás és inzulinadagolás hatását több napon át modellezni, egyéni fiziológiai különbségek figyelembevételével.

\section*{Szimulátor 2: DMMS.R}


A \textbf{DMMS.R} (Diabetes Mellitus Metabolic Simulator -- Research) egy kutatási célokra fejlesztett, professzionális metabolikus szimulátor. Az alkalmazás képes 1-es és 2-es típusú cukorbetegség, valamint a prediabétesz élettani folyamatait modellezni virtuális pácienseken. A szimulátor 55 virtuális alanyt tartalmaz, különböző betegcsoportokra bontva: 1-es típusú gyermek, serdülő és felnőtt, 2-es típusú felnőtt, valamint prediabéteszes felnőtt. Mindegyik alany egyedi, előre megadott anyagcsere-paraméterekkel rendelkezik. A szimulátor célja klinikai kísérletek előzetes és biztonságos támogatása.

\medskip
A szimuláció alapját egy kompartmentalizált glükóz--inzulin anyagcsere modell képezi, amely a hormonok, vércukorszint és az inzulin dinamikáját perces időfelbontásban követi. A modell figyelembe veszi az étkezéseket, a fizikai aktivitást, az inzulinadagolást, a gyógyszeres kezelést és a cirkadián ritmust is. A rendszer támogatja az inzulin farmakokinetikáját és farmakodinamikáját, a biohasznosulást, a lassú felszabadulású készítményeket, valamint az aktív inzulin mennyiségének számítását.

\section{Moduláris felépítés}

A \textbf{DMMS.R moduláris felépítésű, három fő elemből áll:}
\begin{enumerate}
    \item Virtuális páciensek adatbázisa, amely tartalmazza a különböző anyagcsere-jellemzőket.
    \item Matematikai szimulációs motor, amely kiszámítja a fiziológiai folyamatokat.
    \item Grafikus felület, amelyen keresztül konfigurálható a szimuláció.
\end{enumerate}

\section{Konfigurálható szenzorok és vezérlők}

A szimulátor többféle szenzor-, vezérlő- és adagolóelem konfigurálását teszi lehetővé:
\begin{enumerate}
    \item Ideális és zajjal terhelt CGM és SMBG szenzorok.
    \item Étkezési, korrekciós és bázisinzulin vezérlők.
    \item Inzulinpumpák és injekciós adagoló rendszerek.
\end{enumerate}

A programban megadhatók gyógyszeres kezelések, edzésprogramok és étkezések a páciensek számára. A szimuláció során zaj is alkalmazható a valós mérési és adagolási bizonytalanság modellezésére.

\section{Eredmények és felhasználás}

A szimulációk eredményei részletes idősor-grafikonokon, \emph{CVGA} (Control-Variability Grid Analysis) elemzésen, valamint egyéni és populációs szintű statisztikai mutatókon jelennek meg. Az adatok \texttt{CSV} és \texttt{MATLAB} formátumban exportálhatók.

A \textbf{DMMS.R} elsősorban kutatási, fejlesztési és orvostechnikai validációs célokat szolgál. Alkalmas inzulinpumpák, gyógyszeres terápiák és digitális diabétesz-menedzsment rendszerek előzetes vizsgálatára és tesztelésére, anélkül, hogy valós betegeken kellene kísérletezni. A program kifejezetten klinikai kutatási környezetbe készült.


\section*{Szimulátorok összehasonlítása}

\begin{table}[h!]
\centering
\footnotesize
\renewcommand{\arraystretch}{1.3}
\begin{tabularx}{\textwidth}{l X X}
\hline
\textbf{Szempont} & \textbf{py-mGIPsim} & \textbf{DMMS.R} \\
\hline
Fő cél &
\makecell[l]{Oktatási, kutatási és fejlesztési\\célú, rugalmas szimuláció} &
\makecell[l]{Klinikai kutatási és validációs célú,\\ professzionális szimulátor} \\

Modellezett betegek &
\makecell[l]{20 virtuális 1-es típusú\\cukorbeteg} &
\makecell[l]{55 virtuális beteg: 1-es típus, 2-es típus,\\ prediabétesz} \\

Részletesség &
\makecell[l]{Közepes–magas szintű\\fiziológiai modell} &
\makecell[l]{Nagyon részletes, klinikailag orientált\\ kompartment modellek} \\

Felhasználói felület &
\makecell[l]{GUI, interaktív CMD,\\egyszerű CLI} &
\makecell[l]{Dedikált grafikus felület} \\

Terápia / vezérlés &
\makecell[l]{MDI, SAP, Hybrid Closed-loop} &
\makecell[l]{Inzulinpumpák, injekciók, CGM/SGBM,\\ gyógyszerek} \\

Zaj és realitás &
\makecell[l]{Főként determinisztikus \\modellezés} &
\makecell[l]{Mérési és adagolási zaj} \\

Eredmények &
\makecell[l]{Vércukor-időfüggvények, jól\\ vizualizálható kimenet} &
\makecell[l]{Részletes idősorok, statisztikai elemzések} \\

Bővíthetőség &
\makecell[l]{Nagyon jól bővíthető, \\fejlesztőbarát} &
\makecell[l]{Strukturált, klinikai környezethez kötött} \\

Technológia &
\makecell[l]{Python, JIT-optimalizált futás} &
\makecell[l]{Kutatásra fókuszált matematikai motor} \\
\hline
\end{tabularx}
\end{table}



\section*{Források}

\begin{itemize}
    \item \url{https://hu.wikipedia.org/wiki/Cukorbetegs%C3%A9g#1-es_t%C3%ADpus%C3%BA_diabetes_mellitus}
    \item \url{https://hu.wikipedia.org/wiki/Cukorbetegs%C3%A9g#1-es_t%C3%ADpus%C3%BA_cukorbetegség}
\end{itemize}

\end{document}

